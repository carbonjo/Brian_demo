\documentclass{beamer}
\usetheme{Madrid}
\usecolortheme{default}

\title{Riemann Sums}
\subtitle{Approximating the definite integral}
\author{Calculus}
\date{}

\begin{document}

\frame{\titlepage}

\begin{frame}{What is a Riemann sum?}
  \begin{itemize}
    \item Approximates \(\displaystyle\int_a^b f(x)\,dx\)
    \item Divide \([a,b]\) into \(n\) subintervals
    \item Build rectangles; sum their areas
  \end{itemize}
  \[
    S_n = \sum_{i=1}^{n} f(x_i^*)\, \Delta x, \quad \Delta x = \frac{b-a}{n}
  \]
\end{frame}

\begin{frame}{Left, Right, Midpoint}
  \begin{block}{Sample point \(x_i^*\)}
    \begin{itemize}
      \item \textbf{Left:} \(x_i^* = x_{i-1}\)
      \item \textbf{Right:} \(x_i^* = x_i\)
      \item \textbf{Midpoint:} \(x_i^* = \frac{x_{i-1}+x_i}{2}\)
    \end{itemize}
  \end{block}
  Different choices give different approximations; all tend to the same limit as \(n\to\infty\).
\end{frame}

\begin{frame}{Limit = Integral}
  \[
    \int_a^b f(x)\,dx = \lim_{n\to\infty} \sum_{i=1}^{n} f(x_i^*)\, \Delta x
  \]
  \vspace{1em}
  As \(n\) increases, the Riemann sum gets closer to the true area under the curve.
\end{frame}

\begin{frame}{Example: \(f(x)=x^2\) on \([0,2]\)}
  \begin{itemize}
    \item \(n=4\), \(\Delta x = \frac{1}{2}\)
    \item Left sum: \(S_4 \approx 1.75\)
    \item Exact: \(\int_0^2 x^2\,dx = \frac{8}{3} \approx 2.667\)
  \end{itemize}
  \vspace{0.5em}
  Try larger \(n\) (e.g.\ in the notebook or Flask app) to see the approximation improve.
\end{frame}

\begin{frame}{Summary}
  \begin{itemize}
    \item Riemann sum = sum of rectangle areas
    \item Left/right/midpoint = different sample points
    \item Limit as \(n\to\infty\) equals the definite integral
  \end{itemize}
\end{frame}

\end{document}
